\documentclass{article}

\usepackage{hcar}

\begin{document}

% Agda-NA.tex
\begin{hcarentry}[section,updated]{Agda}
\label{agda}
\report{Ulf Norell}%11/13
\status{actively developed}
\participants{Ulf Norell, Nils Anders Danielsson, Andreas Abel,
Jesper Cockx, Makoto Takeyama,
Stevan Andjelkovic, Jean-Philippe Bernardy, James Chapman,
Dominique Devriese, P\'eter Divi\'anszky,
Fredrik Nordvall Forsberg, Olle Fredriksson, Daniel Gustafsson,
Alan Jeffrey, Fredrik Lindblad, Guilhem Moulin, Nicolas Pouillard, Andrés Sicard-Ramírez
and many others}
\makeheader

Agda is a dependently typed functional programming language (developed
using Haskell). A central feature of Agda is inductive families,
i.e., GADTs which can be indexed by \emph{values} and not just types.
The language also supports coinductive types, parameterized modules,
and mixfix operators, and comes with an \emph{interactive}
interface---the type checker can assist you in the development of your
code.

A lot of work remains in order for Agda to become a full-fledged
programming language (good libraries, mature compilers, documentation,
etc.), but already in its current state it can provide lots of fun as
a platform for experiments in dependently typed programming.

Since the release of Agda~2.4.0 in June 2014 a lot has happened in the Agda
project and community. For instance:
\begin{itemize}
\item There have been two Agda courses at the Oregon Programming Languages
Summer School (OPLSS). In 2014 by Ulf Norell, and in 2015 by Peter Dybjer.
\item Agda has moved to github: \url{https://github.com/agda/agda}.
\item Agda~2.4.2 was released in September 2014, and the latest stable
version is Agda~2.4.2.4, released in September 2015.
\item The restriction of Agda to not use Streicher's Axiom K was proved
correct by Jesper Cockx et al. in the ICFP 2014 paper {\em Pattern Matching
without K}.
\item Instance arguments are now powerful enough to emulate Haskell-style
type classes.
\item The reflection machinery has been extended, making it possible to
define convenient reflection based tactics.
\item Improved compiler performance, and a new backend targeting the
Utrecht Haskell Compiler (UHC).
\end{itemize}
Release of Agda~2.4.4 is planned for early 2016.

\FurtherReading
  The Agda Wiki: \url{http://wiki.portal.chalmers.se/agda/}
\end{hcarentry}

\end{document}
