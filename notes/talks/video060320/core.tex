\NeedsTeXFormat{LaTeX2e}
\documentclass[25pt,dvips,fleqn]{foils}
\usepackage[a4paper,landscape,margin=25pt]{geometry}
\usepackage{alltt}
\usepackage[dvips]{color}
\usepackage{proof}
\title{Agda2 Core Language Proposal}
\author{}
% foiltex parameters
\date{2006-03-22}
\MyLogo{}
\leftheader{}
\rightheader{}
\rightfooter{}
% foiltex modification for possibly tighter lineskip.
\makeatletter
\setlength\foilheadskip{-30pt}
\setlength\parskip{18\p@ \@plus 4\p@ \@minus 18\p@}
\renewcommand\@listIa{\leftmargin\leftmargini
\topsep 14\p@ \@plus 2\p@ \@minus 14\p@
\parsep 14\p@ \@plus 4\p@ \@minus 14\p@
\itemsep 14\p@ \@plus 4\p@ \@minus 14\p@}
%
\def\normalsize{\@setfontsize\normalsize\@xxvpt{32}%
\abovedisplayskip 30\p@ \@plus 3\p@ \@minus 30\p@
\belowdisplayskip \abovedisplayskip
\abovedisplayshortskip \z@ \@plus 3\p@
\belowdisplayshortskip 7\p@ \@plus 3\p@ \@minus 7\p@
\let\@listi\@listIa}
%\normalsize
\def\small{\@setfontsize\small\@xxpt\@xxvpt
\abovedisplayskip 30\p@ \@plus 3\p@ \@minus 30\p@
\belowdisplayskip \abovedisplayskip
\abovedisplayshortskip \z@ \@plus 3\p@
\belowdisplayshortskip 7\p@ \@plus 3\p@ \@minus 7\p@
\if@compatibility
 \let\@listi\@listIb
\else
 \let\@listi\@listIa\fi
}
\makeatother
% proof.sty parameter
\inferLineSkip=8pt
% latex parameter
\def\arraystretch{1.2}
% colors
\newcommand\blue{\color{blue}}
\newcommand\red{\color{red}}
\newcommand\gray{\color[gray]{0.4}}

\newcommand{\lbr}{\lbrack\!\lbrack}
\newcommand{\rbr}{\rbrack\!\rbrack}
\newcommand{\sem}[1]{\mbox{$\lbr #1 \rbr$}}

% abbreviations
\newcommand\ar{\mathop{\mathrm{ar}}}
\newcommand\add{\mathop{\mathsf{add}}}
\newcommand\suc{\mathop{\mathsf{suc}}}
\newcommand\Label{\mathop{\mathsf{Label}}}
\newcommand\SN{\mathop{\mathsf{SN}}}
\newcommand\DD{\sf D}
\newcommand\El{\mathop{\mathsf{El}}}
\newcommand\Fun{\mathop{\mathsf{Fun}}}
\newcommand\Sum{\mathop{\mathsf{Sum}}}
\newcommand\E{\mathop{\mathsf{E}}}%{\sf E}
\newcommand\len{\mathop{\mathrm{length}}}
\newcommand\llet{\mathop{\mathrm{let}}}
\newcommand\Nat{\mathsf{Nat}}
\newcommand\Unit{\mathsf{Unit}}
\newcommand\Prop{\mathsf{Prop}}
\newcommand\Prf{\mathsf{Proof}}
\newcommand\Set{\mathsf{Set}}
\newcommand\where{\mathsf{where}}
\newcommand\type{\mathsf{type}}
\newcommand\Kind{\mathsf{Kind}}
\newcommand\data{\mbox{\sffamily\slshape data}}
\newcommand\fun{\mbox{\sffamily\slshape fun}}
\newcommand\with{\mbox{\sffamily\slshape with}}
%\newcommand\where{\mbox{\sffamily\slshape where}}
\newcommand\correct{\mathrm{correct}}
\newcommand\Type{\mathrm{Type}}
\newcommand\lam{}\let\lam\lambda
\newcommand\Down{}\let\Down\Downarrow
\newcommand\down{}\let\down\downarrow
\newcommand\Up{}\let\Up\Uparrow
\newcommand\up{}\let\up\uparrow
\newcommand\hdrs{\mathrel{\succ}^*}
\newcommand\hdr{\mathrel{\succ}}
\newcommand\redQ{{\red ???}}
\newcommand\mbf[1]{\textbf{\slshape #1}} 
\newcommand\Tparam{\mbf T_{\mathrm{param}}}
\newcommand\Tindex{\mbf T_{\mathrm{index}}}
\newcommand\Mparam{\mbf M_{\mathrm{param}}}
\newcommand\Mindex{\mbf M_{\mathrm{index}}}
\begin{document}
%\begin{abstract}
%\end{abstract}
\maketitle
\rm
%\rightfooter{}

\foilhead{Untyped Programming Language}

Untyped $\lambda$-calculus
+ primitive constants $c$ + defined constants $f$
$$
M,N ~::= ~n ~|~ M\,N ~|~ \lambda x.M ~~~~~~ n ::= x ~|~ c ~|~ f
$$
Fixed arities for constants:
\tabular[t]l $\ar(0)=0$, $\ar(\suc)=1$, $\ar(\add)=2$, $\cdots$ \endtabular

\foilhead{Untyped Programming Language}

 We write $p,q,\dots$ for vectors of variables. Each $f$ is defined
by left-linear mutually-disjoint pattern-matching clauses:
$$f~p~(c_i~q_i) = M_i$$

 $\lambda (x,p).M = \lambda x.\lambda p.M$

 $\lambda ().M = M$

 We have also non-recursive abreviations $f=M$

\foilhead{Telescopes}

 $$\Delta~::=~()~|~(x:A,\Delta)$$

 We define the context $p:\Delta$ 

 $(x,p):(x:A,\Delta)$ is $x:A,p:\Delta$

 Telescopes (as types) and vectors (of elements) form a model
of type theory with $\Sigma$-types

 $(M,u):(x:A,\Delta)$ iff $M:A$ and $u:\Delta(x=M)$

\foilhead{Telescopes}

 If $p:\Delta\vdash A$ we define $(p:\Delta)\rightarrow A$ by induction on $\Delta$

 If $\Delta$ is $()$ this is $A$ 

 $((y,q):(y:B,V))\rightarrow A = (y:B)\rightarrow (q:V)\rightarrow A$

 We define $(p:\Delta)\rightarrow U$ by

 $(p:\Delta)\rightarrow () = ()$

 $(p:\Delta)\rightarrow (y:B,V) = (g:(p:\Delta)\rightarrow B,(p:\Delta)\rightarrow V(y=g~p))$

 If $v:(p:\Delta)\rightarrow U$ and $u:\Delta$ then $v~u:U(p=u)$

\foilhead{Telescopes}

 We define $(p:\Delta,V)$ by induction on $\Delta$

 If $\Delta = ()$ this is $V$ and if $((x,q):(x:A,U),V) = (x:A,(q:U,V))$

 We then have $((p,q):(p:\Delta,V))\rightarrow W = (p:\Delta)\rightarrow (q:V)\rightarrow W$

\foilhead{Telescopes}

 If we have $p:\Delta\vdash u:U$ then $\lambda p.u :(p:\Delta)\rightarrow U$ 

 $\lambda p.() = ()$

 $\lambda p.(M,u) = (\lambda p.M,\lambda p.u)$

 We define also $()~v = ()$ and $(M,u)~v = (M~v,u~v)$ 

 We have $(\lambda p.u)~v = u(p=v)$

\foilhead{Terms in $\beta$-normal form}

 $N~::=~\lambda x.N~|~h~u,~~~~~~~h~::=~x|~c~|~d~|~f$

 Type-checking

 $$\frac{\Gamma,x:A\vdash N:B}{\Gamma\vdash \lambda x.N\uparrow (x:A)\rightarrow B}$$

 $$\frac{\Gamma\vdash u\uparrow \Delta~~~~\Gamma\vdash B = A(p=u)}{\Gamma\vdash h~u\uparrow B}
        {~~~h:(p:\Delta)\rightarrow A}$$

 $$\frac{\Gamma\vdash u\uparrow \Delta_i(q=v)}{\Gamma\vdash c_i~u\uparrow d~v}$$

with $d = \lambda q.(c_1~\Delta_1,\dots,c_k~\Delta_k)$

\foilhead{Type-checking}

$$\frac{\Gamma\vdash A~\type _0~~~~~\Gamma,x:A\vdash B~\type _0}
       {\Gamma\vdash (x:A)\rightarrow B~\type _0}$$

$$\frac{\Gamma\vdash N\uparrow \Set}{\Gamma\vdash N~\type _0}$$

\foilhead{Conversion algorithm}

 $$\frac{\Gamma,x:A\vdash M_1~x = M_2~x\uparrow B}{\Gamma\vdash M_1=M_2\uparrow (x:A)\rightarrow B}$$

 $$\frac{\Gamma\vdash u=u'\uparrow \Delta}{\Gamma\vdash h~u=h~u'}~~~{h:\Delta\rightarrow A}$$

 $$\frac{\Gamma\vdash u=u'\uparrow \Delta_i(q=v)}{\Gamma\vdash c_i~u = c_i~u'\uparrow d~v}$$

with $d = \lambda q.(c_1~\Delta_1,\dots,c_k~\Delta_k)$

 $$\frac{\Gamma\vdash M=M'\uparrow A~~~~~\Gamma\vdash u=u'\uparrow \Delta(x=M)}
        {\Gamma\vdash (M,u)=(M',u')\uparrow (x:A,\Delta)}~~~~~~~
   \frac{}{\Gamma\vdash () = ()\uparrow ()}$$

\foilhead{Signature}

 Collection of constants with types $c:A$ and definitions

 We can add

 (1) new constant $c:A$ if $\vdash A$, $c$ fresh name

 (2) new definition for a constant $d:A$ or $f:A$ already declared


 2 kind of definitions: constants for {\em data types} and for 
{\em recursively defined functions} over these data types

\foilhead{Signature}

 Constants for data types

 For $d:\Delta\rightarrow\Set$, and $p:\Delta\vdash \Delta_i~\type _0$

 $d = \lambda p.(c_1~\Delta_1,\dots,c_k~\Delta_k)$

 Typing rules

 $$\frac{}{\Gamma\vdash d:\Delta\rightarrow\Set}$$

 $$\frac{\Gamma\vdash u:\Delta~~~~~\Gamma\vdash v:\Delta_i(p=u)}{\Gamma\vdash c_i~v:d~v}$$


%This definition adds also $c_i:(p:\Delta)\rightarrow \Delta_i\rightarrow c~p$

\foilhead{Signature}

 Constants for recursively defined functions

 If we have $d:U\rightarrow\Set$ 
and $d =  \lambda q.(c_1~\Delta_1,\dots,c_k~\Delta_k)$ and

 $f:(p:\Delta)\rightarrow (x:d~u)\rightarrow A$ 

 Notice that $p:\Delta\vdash u:U$

 We can add the definition

 $f = \lambda p.(c_1~p_1\rightarrow M_1,\dots,c_k~p_k\rightarrow M_k)$

provided $p:\Delta,p_i:\Delta_i(q=u)\vdash M_i:A(x=c_i~p_i)$

\foilhead{Signature}

 New conversion rule

 $$\frac{\Gamma\vdash v:\Delta~~~~~\Gamma\vdash w:\Delta_i(q=u(p=v))}
        {\Gamma\vdash f~v~(c_i~w) = M_i(p=v,p_i=w):A(p=v,x=c_i~w)}$$

 This can be also interpreted as an untyped reduction rule

 $f~v~(c_i~w)\rightarrow M_i(p=v,p_i=w)$

\foilhead{Signature}

 {\bf Example}: universe

 $U:\Set,~T:U\rightarrow\Set$

 We can define $U = (\hat{n},\hat{\pi}~(x:U,f:T~x\rightarrow U))$

 After this, we can define

 $T = (\hat{n}\rightarrow N,\hat{\pi}~x~f\rightarrow \Pi~(T~x)~(\lambda y.T~(f~y)))$

 $T~\hat{n} = N$

 $T~(\hat{\pi}~x~f) = \Pi~(T~x)~(\lambda y.T~(f~y))$

 \foilhead{Type-checking}

$$\frac{\Gamma\vdash A~\type _m~~~~~\Gamma,x:A\vdash B~\type _m}
       {\Gamma\vdash (x:A)\rightarrow B~\type _m}$$

$$\frac{\Gamma\vdash N\uparrow \Set _l~~~l\leq m}
       {\Gamma\vdash N~\type _m}$$

$$\frac{l<m}{\Gamma\vdash \Set _l~\type _m}$$

\foilhead{Where clause}

$$N~::=~\lambda x.N~|~n~u~|~N~\where~D$$

$$D~::=~p:\Delta = u$$

 The simplest typing rule is

 $$\frac{\Gamma\vdash~\Delta~~~~~~\Gamma\vdash u:\Delta~~~~~~\Gamma,p:\Delta\vdash N:A}
        {\Gamma\vdash N~\where~p:\Delta=u~:~A}$$

\foilhead{Example}

 $V:\Set,~a_0:V,~comp:V\rightarrow (V\rightarrow\Set)\rightarrow V,
  ~(\epsilon):V\rightarrow V\rightarrow\Set$

 $h:(a,x:V)\rightarrow (P:V\rightarrow\Set)\rightarrow
         x\epsilon (comp~a~P)\Leftrightarrow x\epsilon a\wedge P~x$

 We can then compute a witness for the telescopes

 $\emptyset:V,h_0:(x:V)\rightarrow \neg x\epsilon \emptyset$

 $\emptyset = comp~ a_0~ (\lambda x.\perp),~~~h_0=\dots$

 $(\cap):V\rightarrow V\rightarrow\Set,
  h_1:(x,y,z:V)\rightarrow x\epsilon y\cap z\Leftrightarrow x\epsilon y\wedge x\epsilon z$

 $(\cap) = \lambda x.\lambda y.comp~x~(\lambda z.z\epsilon y),~h_1=\dots$


\foilhead{Where clause}

 The proof development has the following structure

 $p_1:\Delta_1,p_2:\Delta_2,p_3:\Delta_3,\dots$

 At each level we have $p_1:\Delta_1,\dots,p_{n-1}:\Delta_{n-1}\vdash u_{n}:\Delta_{n}$

 $\Delta_n$ should be an abstract specification of $u_n$

\foilhead{Where clause}

 So far, we have tried (like in Automath $\lambda\Delta$) a more complicated
type-checking for the where-clause

 $p_1:\Delta_1=u_1,\dots$

 The type-checking of $\dots$ depends not only on the {\em type}
of $p_1$ but also on its {\em definition} $u_1$

 It may be that $\dots$ is not correct in the context $p_1:\Delta_1$

\foilhead{Where clause}

 {\bf Example}: give the data type $N_2=(0,1)$

 If we declare $Bool:\Set = N_2$ we don't have

 $Bool:\Set\vdash 0:Bool$

but $(N_2,0)$ is of type $(Bool:\Set,x:Bool)$

 In $\lambda\Delta$ we have

 $Bool:\Set=N_2\vdash 0:Bool$

 One can argue that this is not {\em modular}: further type-checking
depends not only on the types (the interface), but also on the actual
{\em definition}

\foilhead{Example}

 Danko Ilik has formalised Zermelo's 1904 proof of the well-ordering 
theorem in type theory extended with the extensional axiom of choice

 From type theory, one needs $\Pi,\Sigma,N_2,N_1,N_0$ and the function

 $T:N_2\rightarrow\Set$ with $T~0=N_0,~T~1 = N_1$

 It would be very interesting to see if this proof can be represented
with this primitive $\where$ mechanism




\end{document}     
 
